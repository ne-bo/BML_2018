%%
%% Author: natasha
%% 03.10.18
%%


\documentclass[a4paper, 11pt]{article}
%% Language and font encodings
\usepackage[english]{babel}
\usepackage[utf8x]{inputenc}
\usepackage[T1]{fontenc}

%% Sets page size and margins
\usepackage[a4paper,top=3cm,bottom=2cm,left=3cm,right=3cm,marginparwidth=1.75cm]{geometry}

\usepackage{comment} % enables the use of multi-line comments (\ifx \fi)
\usepackage{lipsum} %This package just generates Lorem Ipsum filler text.
\usepackage{fullpage} % changes the margin
%% Useful packages
\usepackage{amsmath}
\usepackage{graphicx}
\usepackage{caption}
\usepackage{subcaption}
\usepackage[colorinlistoftodos]{todonotes}
%\usepackage[colorlinks=true, allcolors=blue]{hyperref}
\captionsetup{compatibility=false}
\usepackage{float}
\usepackage{hyperref}

\begin{document}
    %Header-Make sure you update this information!!!!
    \noindent
    \large\textbf{BML Project: Learning Priors for Adversarial Autoencoders} \hfill\\
    \hfill Teammates: Belozerova Polina, Safin Alexander, Pavlovskaia Natalia \\

    \begin{abstract}
        This work is a reproducing of the \cite{original_paper}.

        Most deep latent factor models choose simple priors for simplicity, tractability or
        not knowing what prior to use. Recent studies show that the choice of the prior
        may have a profound effect on the expressiveness of the model, especially when
        its generative network has limited capacity. In this paper, we propose to learn a
        proper prior from data for adversarial autoencoders (AAEs). We introduce the
        notion of code generators to transform manually selected simple priors into ones
        that can better characterize the data distribution. Experimental results show that
        the proposed model can generate better image quality and learn better disentangled
        representations than AAEs in both supervised and unsupervised settings. Lastly,
        we present its ability to do cross-domain translation in a text-to-image synthesis
        task.
    \end{abstract}

    \section*{Problem Statement}
    We decided to reproduce the following:
    \begin{itemize}
        \item {Unsupervised images generation}
        \item {Supervised images generation}
    \end{itemize}

    \section*{Data}
    The Datasets used:
    \begin{itemize}
        \item {MNIST}
        \item {CIFAR-10}
    \end{itemize}


    \section*{Results}
    \begin{itemize}
        \item {Usupervised images generation

        \centering
        \begin{tabular}{|c|c|c|}
            \hline
            Dataset & Model & Inception score \\\hline


            & Our & ?? \\\cline{2-3}
            MNIST & Original Paper & Unknown\\\cline{2-3}
            \hline
            & Our &??\\\cline{2-3}
            CIFAR & Original Paper & 6.52 \\\cline{2-3}
            \hline
        \end{tabular}
        \centering
        }

        \item {Supervised images generation

        \begin{center}
            \begin{figure}[H]{\textwidth}
                \centring
                \begin{subfigure}[b]{\textwidth}
                    \centering
                    \begin{subfigure}[b]{0.15\textwidth}
                        \includegraphics[width=\linewidth, height=\linewidth]{figures/our_mnist_supervised.png}
                    \end{subfigure}%
                    \begin{subfigure}[b]{0.15\textwidth}
                        \includegraphics[width=\linewidth, height=\linewidth]{figures/paper_mnist_supervised.png}
                    \end{subfigure}%
                    \caption{MNIST}
                \end{subfigure}
                \begin{subfigure}[b]{\textwidth}
                    \centering
                    \begin{subfigure}[b]{0.15\textwidth}
                        \includegraphics[width=\linewidth, height=\linewidth]{figures/our_cifar_supervised.png}
                    \end{subfigure}%
                    \begin{subfigure}[b]{0.15\textwidth}
                        \includegraphics[width=\linewidth, height=\linewidth]{figures/paper_cifar_supervised.png}
                    \end{subfigure}%
                    \caption{CIFAR}
                \end{subfigure}
                \caption{Results for supervised learning the different datasets}
            \end{figure}
        \end{center}
        }
    \end{itemize}


    \section*{Contribution}
    \begin{itemize}
        \item {Belozerova Polina: reading papers, ...}
        \item {Safin Alexander: reading papers, ...}
        \item {Pavlovskaia Natalia: reading papers, code for training procedure, images generation}
    \end{itemize}

    \nocite{*} % Print all references regardless of whether they were cited in the poster or not
    \bibliographystyle{plain} % Plain referencing style
    \bibliography{sample}
    % Use the example bibliography file sample.bib

\end{document}